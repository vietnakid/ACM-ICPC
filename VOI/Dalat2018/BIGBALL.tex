\documentclass[12pt,a4paper,oneside]{article}

\usepackage[utf8]{vietnam}
\usepackage[english]{babel}
\usepackage{format}
\usepackage{verbatim}

\header{\LARGE PreVOI Đà Lạt 2018}

\begin{document}

\problemtitle{BIGBALL}

\renewcommand{\baselinestretch}{1.25}
\setlength{\parskip}{1em}

Thủ môn Đặng Văn Lâm rất giỏi về hình học. Để chuẩn bị tinh thần nhận cúp trong trận chung kết AFF Cup lượt về giữa Việt Nam - Malaysia, huấn luyện viên Park Hang Seo đã đố Văn Lâm một bài toán như sau:

Trong không gian ba chiều có $n$ trái banh hình cầu được đánh số từ $1$ đến $n$, trái banh thứ i được đặt tại điểm có toạ độ ($xi$, $yi$, $zi$) và có bán kính $ri$. Một chiếc xà ngang kéo dài vô tận và đi qua 2 điểm $A$, $B$ phân biệt có toạ độ $A(xA,yA,zA)$ và $B(xB,yB,zB)$, độ to của xà ngang là không đáng kể.

Nhiệm vụ của Xuân Lâm là với mỗi trái banh thứ $i$, tìm ra $ri$ là bán kính nguyên tối thiểu sao cho trái banh thứ $i$ tiếp xúc hoặc cắt xà ngang $AB$. 

Hãy giúp Văn Lâm tìm kết quả của bài toán trên.

\renewcommand{\baselinestretch}{1.0}
\setlength{\parskip}{0.25em}

\heading{Dữ liệu}

\begin{itemize}
\item Dòng đầu tiên chứa số nguyên dương $n$ - là tổng số trái banh.
\item Dòng thứ hai chứa ba số nguyên dương $xA, yA, zA$ - toạ độ của điểm A.
\item Dòng thứ ba chứa ba số nguyên dương $xB, yB, zB$ - toạ độ của điểm B.
\item n dòng tiếp theo, mỗi dòng chứa ba số nguyên $xi, yi, zi$ lần lượt là toạ độ của trái banh thứ i.
\end{itemize}

\heading{Kết quả}

\begin{itemize}
\item Gồm n dòng, mỗi dòng ứng với bán kính nguyên nhỏ nhất của trái banh thứ i để có thể tiếp xúc hoặc cắt xà ngang.
\end{itemize}

\heading{Ví dụ}

\begin{example}
\exmp{%
5 
0 0 0 
1 0 0
100 0 0
5 1 0
1 1 1
1 2 3
100 100 100
}{%
0
1
2
4
142
}%
\exmp{%
5 
5 2 3 
8 7 6
2 8 6
8 5 4
2 3 4
0 0 0
4 5 6
}{%
6
2
4
4
3
}%
\end{example}

\heading{Giới hạn}

\begin{itemize}
\item Subtask 1 (30\%): $1 \leq n \leq 100$; $0 \leq x, y \leq 100$; $z = 0$.
\item Subtask 2 (20\%): $1 \leq n \leq 10^5$; $0 \leq x, y \leq 10^9$; $z = 0$.
\item Subtask 3 (20\%): $1 \leq n \leq 100$; $0 \leq x, y, z \leq 100$.
\item Subtask 4 (30\%): $1 \leq n \leq 10^5$; $0 \leq x, y, z \leq 10^9$.
\end{itemize}

\end{document}
